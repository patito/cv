% LaTeX file for resume 
% This file uses the resume document class (res.cls)

\documentclass[margin]{res}
\usepackage [brazil]{babel}     % nomes e hifenaçã em português
 
\usepackage{t1enc}              % Permite digitar os acentos de forma normal
\usepackage[utf8]{inputenc} 

\topmargin=-0.5in  % start text higher on the page
\setlength{\textheight}{10in} % increase text height to fit resume on 1 page
\begin{document}  
\name{\textit{Paulo Leonardo Benatto}}

\address{Florianopolis,  Brasil \\ benatto@gmail.com \\ +55 (48)99767365 }
                           
                        
\begin{resume}                        
 
\section{Objetivo}	Pleiteio uma vaga de analista de sistemas, com interesse em áreas como redes de 
			computadores, VoIP (\textit{Voice over IP}) e desenvolvimento de softwares (independente de linguagem).
			Atualmente trabalho com desenvolvimento de software utilizando \textit{python} e \textit{javascript}, porém tenho interesse
			em aprender outras tecnologias.
 
\section{Educação}	Universidade Estadual do Oeste do Paraná, Bacharel em Ciência da Computação, Dezembro de 2007
  
\section{Experiência}      

\vspace{-0.1in}
   \begin{tabbing}
   \hspace{2.3in}\= \hspace{1.7in}\= \kill % set up two tab positions
    \textbf{SEC+}    \>\>\textbf{Dez 2012 - Atual}\\
    \textit{Analista de Sistemas}                         
   \end{tabbing}\vspace{-20pt}      % suppress blank line after tabbing
    \vspace{5mm}
    \begin{itemize}
     \item Desenvolvimento em \textit{C ANSI} para análise de \textit{malware} em binários \textit{ELF};
     \item Desenvolvimetno utilizando testes unitários \textit{CUnit};
     \item Documentação de software com \textit{doxygen} e/ou \textit{gtk-doc};
     \item Desenvolvimento \textit{backend python/django} (Intermediário);
     \item Desenvovlimento \textit{JavaScript} (Intermediário);
     \item Utilização de \textit{PostgreSQL};
     \item Utilização de \textit{SCRUM}; 
     \item Administração \textit{Linux};
    \end{itemize}


   \begin{tabbing}
   \hspace{2.3in}\= \hspace{1.7in}\= \kill % set up two tab positions
    \textbf{Digitro Tecnologia - NDS}    \>\>\textbf{Jan 2012 - Dez 2012}\\
    \textit{Analista de Sistemas}                         
   \end{tabbing}\vspace{-20pt}      % suppress blank line after tabbing
    \vspace{5mm}
    \begin{itemize}
     \item Desenvolvimento em \textit{C ANSI} para Interceptação Internet (\textit{TCP/IP});
     \item \textit{Test Driven Development} (TDD) utilizando \textit{CUnit};
     \item Build automático com \textit{JHBuild};
     \item Integração Contínua;
     \item Documentação de software com \textit{doxygen} e/ou \textit{gtk-doc};
     \item Ferramentas de análise estática e dinâmica de código;
     \item Bibliotecas: \textit{GLib}, \textit{PCAP}, \textit{Iconv}, entre outras; 
     \item Utilização de \textit{SCRUM}; 
     \item Administração Linux: \textit{Arch Linux}, \textit{Debian}, \textit{Ubuntu} e \textit{CentOS};
    \end{itemize}
 
   \begin{tabbing}
   \hspace{2.3in}\= \hspace{1.5in}\= \kill % set up two tab positions
    \textbf{Digitro Tecnologia - STE}    \>\>\textbf{Set 2008 - Dez 2011}\\
    \textit{Analista de Sistemas}                         
   \end{tabbing}\vspace{-20pt}      % suppress blank line after tabbing
    \vspace{5mm}
    \begin{itemize}
      \item Instalação e configuração do \textit{FreeSWITCH};
      \item Instalação e configuração \textit{OpenSER} 
      \item Desenvolvimento em equipe de um softphone em liguagem C (\textit{VisualStudio}); 
      \item Desenvolvimetno em equipe de um telefone IP; 
      \item Realização de testes SIP utilizando SIPP;  
      \item Administração de servidores Linux; 
      \item Ferramentas de desenvolvimento: \textit{VI, GCC, make, cmake, VisualStudio, Splint, Valgrind, gdb e subversion};
      \item Banco de dados orientado a documentos: \textit{mongodb}; 
      \item Automatização de serviços utilizando \textit{shell} e \textit{python};
    \end{itemize}

   \begin{tabbing}
   \hspace{2.3in}\= \hspace{1.5in}\= \kill % set up two tab positions
    \textbf{V.Office}    \>\>\textbf{Jan 2008 - Ago 2008}\\
    \textit{Programador Jr}                         
   \end{tabbing}\vspace{-20pt}      % suppress blank line after tabbing
    \vspace{5mm}
    \begin{itemize}
      \item Instalação e Configuração do \textit{PBX IP Asterisk};
      \item Utilização de linguagens de programação (shell script, php, html e css) e banco de dados (MySQL) para o 
	    desenvolvimento de soluções na área de telecomunicações; 
      \item Desenvolvimento de um material sobre Asterisk Avançado utilizando a ferramenta Latex; 
      \item Suporte a clientes VoIP, sempre utilizando o sistema operacional GNU/Linux;
    \end{itemize}

\section{Habilidades}  \textit{Sistemas Operacionais}:  Linux, Windows NT/XP/Vista/7, algum conhecimento em *BSD;

			\textit{Redes}:  Suíte de protocolos TCP/IP (IP, ICMP, TCP, UDP, HTTP, FTP, SMTP, SNMP, e outros) e o modelo ISO/OSI;
  
			\textit{Linguagens de Programação}: C ANSI, Pascal, lua, algum conhecimento em Java / C++ / Python / Ruby / PHP;
  
			\textit{Virtualização}: VirtualBox, VMWare e algum conhecimento em Xen;

			\textit{Administração de Sistemas}:  Análise, investigação e resolução de problemas gerais, planejamento de capacidade e ajustes de performance;

			\textit{Idiomas}: Fluente em Português, Intermediário em Inglês e Espanhol;

 


\section{Atividades}
                Participante ativo do grupo GNOIA que tem como objetivo o auxílio na implantação de Software Livre nas Universidades da região Oeste do Paraná, em especial na Universidade Estadual do Oeste do Paraná (UNIOESTE), instituição onde maior parte dos integrantes do grupo é formado. Além disso, o GNOIA é responsável pela organização de diversos eventos na região de Foz do Iguaçu, contribuindo com palestras, mini-cursos e Install Fest's.
 
\section{Projetos}
		Tenho interesses em desenvolvimento de software livre e sempre que possível tenho um projeto no github (https://github.com/patito/).
		
		\begin{itemize}
		\vspace{2mm}
		\item \textbf{libpenetra}: The libpenetra was created with the goal of studying the windows binary format known as Portable Executable (PE).
		                     With libpenetra you can access all info about PE binaries. (https://github.com/patito/libpenetra)
		\vspace{2mm}                    
		\item \textbf{libmalelf}: The libmalelf is an evil library that SHOULD be used for good! It was developed with the intent to assist in the process
		                     of infecting binaries and provide a safe way to analyze malwares. (https://github.com/SecPlus/libmalelf)\vspace{2mm}
		\item \textbf{malelf}: Malelf is a tool that uses libmalelf to dissect and infect ELF binary. (https://github.com/SecPlus/malelf)
		\end{itemize}
 
\end{resume} 
\end{document}









