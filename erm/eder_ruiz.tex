% LaTeX file for resume 
% This file uses the resume document class (res.cls)
% Template from Internet

\documentclass[margin]{eder}
\usepackage [brazil]{babel}
 
\usepackage{t1enc}
\usepackage[utf8]{inputenc} 

\topmargin=-0.5in
\setlength{\textheight}{10in}
\begin{document}  
\name{\textit{Eder Ruiz Maria}}

\address{Manaus, BR \\ eder.rm82@gmail.com \\ Phone: 069 242424 }
                                   
\begin{resume}                        
 
\section{Summary}    

I'm passionate about understanding how things work and what fascinates me most is the world of computing. This fascination quickly led me to the world of Linux and Open Source which in those days meant knowing how to use the command line to operate the system.

At university Pascal was the first language I learnt, but C quickly became my favourite programming language. Since then I have had experience with several other languages and paradigms, including Delphi, Pascal, C, Assembly x86, C++, Java, Shell Script, Python, PHP, JavaScript, Prolog, Awk, Lua, Wiring, Matlab, Verilog and HDL. I know some of them deeply, others just superficially. Among the languages that I don't know, some I hope not to need, others I can't wait for the contact, such as R and Go.

My professional experience is very strong in Linux, Embedded Linux, C and microcontrollers. Also I've worked with web applications and mobile (Android). I recently had contact with FPGA, GPU, image processing and computer vision. Topics that also interest me are Operating Systems, Real Time Systems, Distributed Systems, High Availability, Fault Tolerance, Data Science and Cloud Computing.

Feel free to get in touch and to look my detailed professional experiences below. You can find my open source projects at my github account in https://github.com/ederrm.
 
\section{Education}	State University of West Paraná, BSc in Computer Science, December 2009.\\ \\
                    Federal University of Santa Catarina, MSc in Automation Engineering, Discipline Distributed System, April 2010.
  
\section{Experience}

\vspace{-0.1in}
  \begin{tabbing}
    \hspace{2.3in}\= \hspace{1.7in}\= \kill
    \textbf{Nokia Institute of Technology}    \>\>\textbf{Jun 2012 - Jun 2015}\\
    \textit{Software Engineer}\\
    \textbf{Main Technologies}: Linux, C/C++, Raspberry PI, Arduino, Java, JavaScript, BlueZ;
  \end{tabbing}\vspace{-20pt} 
\vspace{6mm}

\begin{itemize}
  \item Contributor in the Official Linux Bluetooth protocol stack;
  \item Android development (internal e apps);
  \item Embedded development (Arduino and Respberry PI);
  \item Web development (webpy, AngularJS, SQLite, RESTFul API)
  \item C++ Development using QT Library (Desktop and Meego);
  \item Distributed version control system (git);
  \item Code review tools (phabricator, gerrit);
  \item Iterative and incremental agile software development methodology (Scrum);
  \item CUDA, Matlab, Image Processing, Computer Vision;
\end{itemize}

\vspace{-0.1in}
  \begin{tabbing}
    \hspace{2.3in}\= \hspace{1.7in}\= \kill
    \textbf{Ahgora Systems}    \>\>\textbf{Jan 2012 - Jun 2012}\\
    \textit{Software Engineer}\\        
    \textbf{Main Technologies}: Linux, C, Lua, git, Device Driver, Kernel, Embedded Softwares;
  \end{tabbing}\vspace{-20pt} 
\vspace{6mm}

\begin{itemize}
  \item Device drivers development to Linux Kernel ARM (atmel at91sam9g20);
  \item Development of bate-metal firmware to 8051 using ultra low-power (nordic nrf24le1)
  \item Development of applications to Embedded Linux (C, lua bind)
  \item Linux development in general using C, shell script;
  \item Android development (usingadb and IOIO);
  \item Firmware development to keyboard control (quantum qt60248)
  \item Firmware development to ARM Cortex-M3 (nxp lpc1768)
\end{itemize}

\vspace{2mm}
  \begin{tabbing}
    \hspace{2.3in}\= \hspace{1.7in}\= \kill
    \textbf{Digitro Technology}    \>\>\textbf{Nov 2009 - Dec 2011}\\
    \textit{Software Engineer}\\   
    \textbf{Main Technologies}: Linux, C/C++, Kernel, Devices Driver, Embedded Software;
  \end{tabbing}\vspace{-20pt}
\vspace{6mm}

\begin{itemize}
  \item Device driver development to Linux Kernel (vortex86mx southbridge gpio, omap5912);
  \item Device driver development to uClinux (blackfin 536/537);
  \item Development and customize Bootloader (U-Boot);
  \item Bare-metal firmware development (HCS08)
  \item Development of applications to Embedded Linux (C, Socket and GLib);
  \item Development different kind of application in Linux environment (C, Python, shell);
  \item Subversion Version Control;
\end{itemize}
   
\vspace{2mm}
   \begin{tabbing}
   \hspace{2.3in}\= \hspace{1.5in}\= \kill
    \textbf{Institute of Applied Technology and Innovation}    \>\>\textbf{Feb 2009 - Oct 2009}\\
    \textit{Software Engineer}\\   
    \textbf{Main Technologies}: Linux, C/C++, PIC, Modbus/RTU, RTAI, Java/JNI, beagleboard;
   \end{tabbing}\vspace{-20pt}
\vspace{6mm}
    
\begin{itemize}
  \item Data acquisition using drivers Comedi to national board PCI6143;
  \item Tasks hard real time using RTAI;
  \item Development of a serial communication protocol between PC and Microcontroller PIC;
  \item Application development Modbus/RTU Master;
  \item Development of an application to comunicate to GPS Garmin 18x USB;
  \item Integration of native code with Java, using JNI;
  \item Embedded Linux using beagleboard (OMAP);
\end{itemize}

\section{Skills Base} 

  \textit{Operating System}: Linux, Windows NT/XP/Vista/7 and OSX;

	\textit{Networking}: TCP/IP protocol suite;
  
	\textit{Progamming Languages}: C/C++, Lua, PHP, Python, JavaScript, plus some experience with Java;
  
	\textit{Virtualization}: VirtualBox, VMWare, plus some experience with Xen;

	\textit{Languages}: Fluent in Portuguese, Intermediate in English and Spanish;
 
\section{Open Source Projects}
	\begin{itemize}
		\vspace{2mm}

		\item \textbf{BlueZ}: BlueZ is the Bluetooth stack for Linux kernel-based family of operating systems. Its goal is to program an implementation of the Bluetooth wireless standards specifications for Linux. 

	\end{itemize}
 
\section{More Info}
    \begin{itemize}
        \item \textbf{Linkedin}: https://www.linkedin.com/in/ederrm
         \item \textbf{Github}: https://github.com/ederrm
    \end{itemize}

\end{resume} 
\end{document}









